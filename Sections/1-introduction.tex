\setcounter{page}{0}
\chapter{\chapternameintro}

  O período de entrega de dissertações e teses é caótico e no caminho surgem muitas dúvidas: como faço pra depositar? Quais documentos preciso preparar? Onde imprimir a tese? Existem outros prazos que eu deva ficar atento?

  Com o objetivo de ajudar quem estiver nessa etapa, resolvemos criar este documento que, além de servir como um template não oficial para as teses do IAG, também serve como um guia geral.

  \section{Dicas gerais}

    Quando você estiver escrevendo o texto, deve ficar atento aos capítulos que são obrigatórios, seguindo as normas do IAG. Você pode encotrá-las aqui: \url{https://leginf.usp.br/?resolucao=resolucao-copgr-no-7882-de-25-de-novembro-de-2019}. A parte do texto que diz o que é necessário para realizar o depósito está na seção ``XI – PROCEDIMENTOS PARA DEPÓSITO DA DISSERTAÇÃO/TESE''.

    Resumidamente, a dissertação/tese deve conter: Capa, folha de rosto, resumo em português, resumo em inglês, a lista de figuras, ilustrações, tabelas e acrônimos, introdução, metodologia, resultados, conclusões, perspectivas, bibliografia e, opcionalmente, apêndices e anexos.

    Para a tese de doutorado, você também tem a opção de fazer uma coletânea de artigos. Neste caso é necessário ter pelo menos um artigo submetido e/ou publicado e, para poder utilizá-lo na tese, é preciso ter a autorização da(s) editora(s) e dos co-autores. Então você deve incluir um capítulo após a introdução descrevendo a relação entre os artigos e a tese. É possível misturar capítulos ``normais'' e de artigos para construir uma tese coerente.

    O processo de depósito consiste em entregar uma cópia impressa da dissertação/tese na coordenação do programa. Já a manifestação do orientador dizendo que você está apto(a) a defender, o formulário de sugestão da banca, e o comprovante de artigo publicado (no caso do doutorado) devem ser incluídos no depósito eletrônico, realizado na plataforma Janus.

    No Janus, depois de fazer login, você deve ir em ``Aluno regular'' > ``Depósito''. Lá você terá que preencher algumas informações como seu nome (no formato que aparece em citações), o seu ORCID, e anexar os formulários descritos acima, a tese, e o \textbf{protocolo de recebimento de depósito} (Figura \ref{fig:protocolo}), que será feito quando você depositar o exemplar impresso. Fique atento pois você precisará colocar o título, o resumo, e as palavras-chave do seu trabalho em português e inglês, independente de qual é o idioma no qual você escreveu a tese. Além disso, as palavras-chave podem ter no máximo 150 caracteres, e o resumo não pode passar do limite de 5000 caracteres.
    %
    \begin{figure}[h]
      \centering
      \includegraphics[width=\linewidth]{protocolo_recebimento.jpeg}
      \caption{Foto do protocolo de recebimento de depósito, que você precisa escanear e colocar no Janus.}
      \label{fig:protocolo}
    \end{figure}

    Para imprimir a tese, você pode aproveitar a parceria que o IAG tem com a gráfica do IME. Você só precisa mandar um email para a CPG (\url{ccpastro iag.usp.br}) pedindo autorização. Quando ela for dada, é só encaminhar o email para a Cida, que fará a solicitação junto à grafica. Quando a impressão estiver pronta, depois de um ou dois dias, ela irá te avisar para você poder fazer a retirada.

    Tanto o formulário de sugestão da banca quanto a carta de manifestação do orientador estão disponíveis na seção de formulários do site do IAG (\url{https://www.iag.usp.br/pos-graduacao/formularios}), na parte ``8 - Defesa de Dissertações e Teses''

  % The goal of this document is to provide an unofficial dissertation/thesis template for IAG/USP.

  % \section{General tips for when writing a dissertation/thesis}

  % When writing the text, you should be aware of the IAG/USP norms\footnote{\url{https://leginf.usp.br/?resolucao=resolucao-copgr-no-7882-de-25-de-novembro-de-2019}}. There are some obligatory chapters that should be in the text, the list is shown in ``XI – PROCEDIMENTOS PARA DEPÓSITO DA DISSERTAÇÃO/TESE''.

  % In order to make the deposit, you should have:
  % %
  % \begin{itemize}
  %   \item For PhD: a printed copy of the text, the forms containing the committee suggestion, a letter from the supervisor saying that you are fit for the defense, and proof of one published paper, where you are the first or second co-author, in a refereed international journal. You also need to make a deposit in Janus.
  %   \item For master's: 
  % \end{itemize}

  % IAG has a partnership with the IME's print shop. To print your dissertation/thesis with them you should send an email to \url{ccpastro iag.usp.br} with copy to \url{cida.coelho@iag.usp.br} asking for permission, with the PDF of the text attached. If approved, the text will be sent to the print shop and you will be notified when it is ready (within 1 or 2 days).

  % The forms for the committee and the supervisor manifestation can be found at IAG's website, in the ``Forms'' page: \url{https://www.iag.usp.br/pos-graduacao/formularios}. I recommend that you start thinking about the names that should be in your defense two weeks before making the deposit, and sending each member an email one week before, asking if they accept that their names be suggested.

  % \section{Features of this template}

  %   This template is based on the \ttt{WebLatex}\footnote{\url{https://github.com/sanjib-sen/WebLaTex}} template, created to replace Overleaf with Github, with some modifications. See the link for further details and the advantages of this approach.

  %   We also included the GitDoc extension, which automatically makes commit+push to a repository. To use it, you should enable the extension using \ttt{ctrl+shift+p} > ``GitDoc Enable''
  
  %   \subsection{Acronyms}
  %     The acronyms in this model are handled by the \ttt{acro} package, where you need to defined the acronyms beforehand, in the ``Sections/0.2-list\_of\_acronyms.tex'', using the format:

  %     \begin{lstlisting}[autogobble]
  %       \DeclareAcronym{acronym}{
  %         short = short name,
  %         long  = long name,
  %         cite  = citation %optional
  %       }
  %   \end{lstlisting}

  %   Using this package, the first reference to an acronym is written normally, with the reference if you defined the acronym with one, using the ``long'' name: \ac{splus}. Whenever you make another reference to this acronym, it will use the ``short'' name: \ac{splus}.
    
  %   You can also define acronyms using math-mode: \ac{photoz}. If you use the acronym only once, it will be printed using the long form only, without diplaying the short form, in this case, use can use the \ttt{\textbackslash ac\{acro\}} command followed by  \ttt{\textbackslash acuse\{acro\}}: \ac{specz}\acuse{specz}.

  %   There are some variations on how the acronyms can be printed, such as:
  %   \begin{itemize}
  %     \item Plural: \ttt{\textbackslash acp\{photoz\}} $\rightarrow$ \acp{photoz}
  %     \item Force long name: \ttt{\textbackslash acl\{des\}} $\rightarrow$ \acl{des}
  %     \item Force short name: \ttt{\textbackslash acs\{vhs\}} $\rightarrow$ \acs{vhs}
  %     \item Capital first letter: \ttt{\textbackslash Ac\{photoz\}} $\rightarrow$ \Ac{photoz}
  %   \end{itemize}

  %   \subsection{Citations}
  %     In this template, the citations are handled by the \ttt{natbib} package, with is defined by the commands in the ``Sections/6-bibliography.tex'' file. There, the list of used references are imported from the ``Sections/reference\_list.bib'' file.

  %     This package supports different types of citations, all of them explained here: \url{https://gking.harvard.edu/files/natnotes2.pdf}

  %   \subsection{Examples of tables, figures, listings...}

  %   \subsection{Making figures with \ttt{Matplotlib}}